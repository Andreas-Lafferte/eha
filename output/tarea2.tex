% Options for packages loaded elsewhere
\PassOptionsToPackage{unicode}{hyperref}
\PassOptionsToPackage{hyphens}{url}
\PassOptionsToPackage{dvipsnames,svgnames,x11names}{xcolor}
%
\documentclass[
  12pt,
]{article}
\usepackage{amsmath,amssymb}
\usepackage{setspace}
\usepackage{iftex}
\ifPDFTeX
  \usepackage[T1]{fontenc}
  \usepackage[utf8]{inputenc}
  \usepackage{textcomp} % provide euro and other symbols
\else % if luatex or xetex
  \usepackage{unicode-math} % this also loads fontspec
  \defaultfontfeatures{Scale=MatchLowercase}
  \defaultfontfeatures[\rmfamily]{Ligatures=TeX,Scale=1}
\fi
\usepackage{lmodern}
\ifPDFTeX\else
  % xetex/luatex font selection
\fi
% Use upquote if available, for straight quotes in verbatim environments
\IfFileExists{upquote.sty}{\usepackage{upquote}}{}
\IfFileExists{microtype.sty}{% use microtype if available
  \usepackage[]{microtype}
  \UseMicrotypeSet[protrusion]{basicmath} % disable protrusion for tt fonts
}{}
\makeatletter
\@ifundefined{KOMAClassName}{% if non-KOMA class
  \IfFileExists{parskip.sty}{%
    \usepackage{parskip}
  }{% else
    \setlength{\parindent}{0pt}
    \setlength{\parskip}{6pt plus 2pt minus 1pt}}
}{% if KOMA class
  \KOMAoptions{parskip=half}}
\makeatother
\usepackage{xcolor}
\usepackage[margin=2cm]{geometry}
\usepackage{graphicx}
\makeatletter
\def\maxwidth{\ifdim\Gin@nat@width>\linewidth\linewidth\else\Gin@nat@width\fi}
\def\maxheight{\ifdim\Gin@nat@height>\textheight\textheight\else\Gin@nat@height\fi}
\makeatother
% Scale images if necessary, so that they will not overflow the page
% margins by default, and it is still possible to overwrite the defaults
% using explicit options in \includegraphics[width, height, ...]{}
\setkeys{Gin}{width=\maxwidth,height=\maxheight,keepaspectratio}
% Set default figure placement to htbp
\makeatletter
\def\fps@figure{htbp}
\makeatother
\setlength{\emergencystretch}{3em} % prevent overfull lines
\providecommand{\tightlist}{%
  \setlength{\itemsep}{0pt}\setlength{\parskip}{0pt}}
\setcounter{secnumdepth}{-\maxdimen} % remove section numbering
\ifLuaTeX
\usepackage[bidi=basic]{babel}
\else
\usepackage[bidi=default]{babel}
\fi
\babelprovide[main,import]{spanish}
% get rid of language-specific shorthands (see #6817):
\let\LanguageShortHands\languageshorthands
\def\languageshorthands#1{}
\usepackage{graphicx}
\usepackage{times}
\usepackage{caption}
\usepackage{graphicx}
\usepackage{floatrow}
\floatsetup[figure]{capposition=top}
\floatsetup[table]{capposition=top}
\floatplacement{figure}{H}
\floatplacement{table}{h}
\usepackage{booktabs}
\usepackage{longtable}
\newcommand{\sectionbreak}{\clearpage}
\usepackage{array}
\usepackage{multirow}
\usepackage{wrapfig}
\usepackage{colortbl}
\usepackage{pdflscape}
\usepackage{tabu}
\usepackage{threeparttable}
\usepackage{graphicx}
\usepackage{fancyhdr}
\usepackage{eso-pic}
\AtBeginDocument{\AddToShipoutPictureBG*{\put(0,\dimexpr\paperheight-4cm){\makebox[0pt][l]{\hspace{1cm}\includegraphics[width=4cm]{logo_isuc.png}}}}}
\usepackage[subrefformat=simple]{subfig}
\ifLuaTeX
  \usepackage{selnolig}  % disable illegal ligatures
\fi
\IfFileExists{bookmark.sty}{\usepackage{bookmark}}{\usepackage{hyperref}}
\IfFileExists{xurl.sty}{\usepackage{xurl}}{} % add URL line breaks if available
\urlstyle{same}
\hypersetup{
  pdftitle={Tarea 2},
  pdflang={es},
  colorlinks=true,
  linkcolor={blue},
  filecolor={Maroon},
  citecolor={Blue},
  urlcolor={blue},
  pdfcreator={LaTeX via pandoc}}

\title{\textbf{Tarea 2}}
\author{Clase: Análisis de Historias de Eventos\\
Profesora: Viviana Salinas\\
Estudiante: Andreas Laffert\\
email:
\href{mailto:alaffertt@estudiante.uc.cl}{\nolinkurl{alaffertt@estudiante.uc.cl}}}
\date{Fecha: 02-08-2023}

\begin{document}
\maketitle

\setstretch{1.15}
\begin{verbatim}
## [1] 0.05164612
\end{verbatim}

\begin{verbatim}
## [1] 0.04828522
\end{verbatim}

\begin{verbatim}
## [1] 0.04079918
\end{verbatim}

\begin{verbatim}
## [1] 0.03741774
\end{verbatim}

\hypertarget{complete-las-tablas-de-vida-completa-para-cada-raza-por-separado-calculando-nmx-tx-y-ex.}{%
\subsubsection{1. Complete las tablas de vida completa para cada raza
por separado, calculando nmx , Tx y
ex.}\label{complete-las-tablas-de-vida-completa-para-cada-raza-por-separado-calculando-nmx-tx-y-ex.}}

Respuesta en archivo Excel adjunto.

\hypertarget{compare-la-experiencia-de-mortalidad-de-estos-cuatro-grupos.-describa-las-diferencias-en-tuxe9rminos-de-nmx-lx-y-ex.-cuxf3mo-cambian-esas-diferencias-a-medida-que-los-cuatro-grupos-avanzan-en-edad}{%
\subsubsection{2. Compare la experiencia de mortalidad de estos cuatro
grupos. Describa las diferencias en términos de nmx, lx y ex. ¿Cómo
cambian esas diferencias a medida que los cuatro grupos avanzan en
edad?}\label{compare-la-experiencia-de-mortalidad-de-estos-cuatro-grupos.-describa-las-diferencias-en-tuxe9rminos-de-nmx-lx-y-ex.-cuxf3mo-cambian-esas-diferencias-a-medida-que-los-cuatro-grupos-avanzan-en-edad}}

Considere en su discusión:

\begin{enumerate}
\def\labelenumi{\alph{enumi})}
\tightlist
\item
  el efecto de la raza y el género en el proceso bajo estudio
\item
  los años en que la tasa de mortalidad alcanza sus valores más altos y
\item
  las diferencias en la función de sobrevivencia. Use gráficos cuando le
  parezca apropiado
\end{enumerate}

\begin{itemize}
\tightlist
\item
  Tendencia transversal a los grupos: a mayor edad, mayor es la tasa de
  mortalidad, lo cual indica una tendencia monótona de la tasa.
  Asimismo, para todos los grupos la tasa de mortalidad alcanza sus
  mayores valores en la edad culmine de la tabla de vida, esto es, la
  población que tiene 100 años o más. Esto sugiere que la probabilidad
  de experimentar la muerte es mucho mayor en dicha edad en comparación
  a edades menores. Por otra parte, es posible sostener que la tasa de
  mortalidad comienza a crecer exponencialmente cuando se adentra en la
  población de 70 años hacía adelante, indicando de algun modo una
  descripción o comportamiento general de la población (a los 80 años o
  más hay la tasa de mortalidad crece considerablemente).
\end{itemize}

Comparando el panel A con el B, es posible apreciar diferencias en la
tasa de mortalidad. Lo primero a mencionar es que la tasa de mortalidad
es generalmente mas baja para los hombres negros que para los hombres
blancos, ya que el valor de esta tasa para la edad maxima en los
primeros es de 0.4 y la de los segundos 0.5 aprox. Segundo, tambien cabe
destacar que la tasa de mortalidad en los recien nacidos (es decir, de
los 0-1 años) es mayor en los hombres negros que en los blancos.
Tercero, el aumento exponencial de la tasa de mortalidad en los hombres
blancos ocurre de manera más brusca o rapida entre los 80 y 90 años. Por
su parte, si bien los hombres negros muestran una tasa relativamente
mayor desde los 60 años, el crecimiento de esta curva es más suave
acercandose a los 80 y 90 años. En definitiva, esto puede sugerir en los
hombres blancos el transito desde los 85 años en adelante incrementa en
mayor medida su probabilidad de morir que en los hombres negros.

Comparando el panel C con D. Al igual que en los hombres, entre las
muejres blanchas y negras se aprecia que estas ultimas tienen tasas de
mortalidad menores que las primeras. Similramente, la tasa de mortalidad
en las muejres negras es mayor en los recien nacios (0 a 1 año) en
comparacion a las blancas. En cuanto a la curva, en las mujeres blancas
esta muestra un crecimiento más acelerado desde los 85 años, ya que si
ene sta edad tenia un avlor aprox de 0.1, en los 95 llega a los 0.25.
Por su lado, las mujeres negras muestran un crecimiento mas suave de la
curva en las edades mayores; si a los 85 rondaba en el 0.1 aprox, en los
95 llega al 0.2. Esta diferencia sugieren que no solo las mujeres
blancas tienen, en general, una mayor tasa de mortalidad que las negras,
si no que ademas esta tasa comienza a incrementarse en mayor medida
antes en las muejres blancjas que en las negras.

Dentro de los blancos. Hombres tienen mayor yasa de mortalidad que las
mujeres en general y a lo largo de la serie. Por ehemplo, en los 90 años
los hombres tienen una tasa de 0.2 y las mujeres del 0.15

Dentro de los negros: Homres tienen mayor tasa de mortalidad que las
mujeres en general y a lo largo de la serie tambien. Por ejemplo en los
60 estos tienen yuna tasa relativamente maor a las mujeres. En los 95
hombres estan en 0.25 y las mujeres en el 0.2

exp

En general, la esperanza de vida para los grupos es de 80 años, pues en
esa edad la esp ya se reduce a 10. En el mayor grupo de edad, la
esperanza de vida ya es minima, cercana a 0. La surva tamn es monotona:
a mayor edad, menor esperanza de vida. La esperanza de vida standar es
la de 80 años.

Ahora, hay algunas diferencias:. La esperanza de vida en los hombres
negros es menor con comparacion a los hombres blancos al comienzo del
ciclo de vida (0-1 año), lo cual se asocia a una mayor tasa de
mortalidad en estos años en los negros. En egeneral, los blancs tienen
mayor esperanza de vida en la serrue

En las mujeres, las blancas tamb tienen mayor esp al comienzo, aunque
menos marcado que en los hombres

Entre hombres y mujeres. Incialmente la esp de vida es mayor para las
mujeres blancas que para los hombres blancos. Esto se mantiene
relativamente estable hasta os 65, ya que en los 50 por ejemplo, los
hombres tiene una esp de 30 y las mujeres de 35. En los negros:
notablemente las mujeres tienen mayor esperanza de vida que los hombres
a la edad inicial, a los 50 tamb se mantiene una dif de 5 años. En
edades avanzadas las esp de tienen a coincidir, es muy baja.

lx set de riesgo

En general se aprecia que el set de riesgo disminuye con la edad, es
decir, hay menos personas a medida que pasa la edad pues experimentan el
evento de morir.

A con B. Se observa que hay un lx mayor en A que en B, y que esta lytma
myesra una caida mas pronunciada que A. Por ejemplo, en los 70 la caida
en B es más acalereda, en donde habian 6 mil negros, los blancos eran 7
mil 500 aprox.

C con D. Lo mismo, hay mas blancas que negras. Tamb hay una pendiente
mas acalerada en las negras. Cuiriso es que al final de la serie, hay
mas negras que blancas, lo que puede asociarse a la menor tasa de
mortalidad de las primeras respecto a las segundas.

Hombres y mujeres. Sean negros o blancos, hay mas mujeres que hombres en
el tiempo. Por ejmplo, en los 80 años. La comparacion de las cirvas da a
entender aquello, a mayor edad aun se mantiene mayor set de riesgo en
las mujeres que en los hombres, y luegi esto cae.

sobrevivencia

Todas las funciones de sobrevivencia muestran una forma comun: una
monotna tendencia al no incremento en funcion del tiempo. Al incio del
tiempo, adquiere valor 1 pues hay alta probabilidad de sobrevivir al
evento. La fncion de sovre cae lentamete, y a mayor edad hay menor porb
de sobrevivir al evento

Esta probabiliad de sobrevivir al evento es ligeramente menor en los B
que en A (relacionado a su mayor tasa de mortalidad en los negros en el
inicio). La funcion de sobre es mayor en A que en B, ya que en los 80
años los blancos tienen un 50\% de sobrevivir mientras qye los negros un
35\%. De todas maneras, tengo mas lx de blancos que de negros.

En las C tambien la f de sobreviencia es mayor que en D. Por ejmplo, a
los 80 las blancas ienen un 60 de sobrevivir, mientras que las negras un
55\% aprox.

Las mujeres tienen mayor sobrevivencia que los hombres.

\begin{figure}

{\centering \includegraphics{C:\Users\alaffert\OneDrive - Ministerio del Trabajo y Previsión Social\Documentos\GitHub\eha\output\tarea2_files/figure-latex/figura1-1} 

}

\caption{Tasa de mortalidad para cada grupo, EE.UU 2017}\label{fig:figura1}
\end{figure}

\begin{figure}

{\centering \includegraphics{C:\Users\alaffert\OneDrive - Ministerio del Trabajo y Previsión Social\Documentos\GitHub\eha\output\tarea2_files/figure-latex/figura2-1} 

}

\caption{Número de supervivientes para cada grupo, EE.UU 2017}\label{fig:figura2}
\end{figure}

\begin{figure}

{\centering \includegraphics{C:\Users\alaffert\OneDrive - Ministerio del Trabajo y Previsión Social\Documentos\GitHub\eha\output\tarea2_files/figure-latex/figura3-1} 

}

\caption{Esperanza de vida para cada grupo, EE.UU 2017}\label{fig:figura3}
\end{figure}

\begin{figure}

{\centering \includegraphics{C:\Users\alaffert\OneDrive - Ministerio del Trabajo y Previsión Social\Documentos\GitHub\eha\output\tarea2_files/figure-latex/figura4-1} 

}

\caption{Función de sobrevivencia para cada grupo, EE.UU 2017}\label{fig:figura4}
\end{figure}

\newpage

\hypertarget{aunque-ha-calculado-tablas-de-vida-especificas-por-raza-y-guxe9nero-piense-cuxf3mo-se-podruxeda-estimar-el-efecto-de-la-raza-sobre-el-evento-morir-usando-una-funciuxf3n-de-la-tabla-de-vida-es-decir-cuuxe1l-es-el-riesgo-de-muerte-entre-los-hombres-si-prefiere-mujeres-blancos-en-comparaciuxf3n-a-los-hombres-mujeres-negros-se-puede-estimar-ese-efecto-con-estos-datos-cuxf3mo-lo-haruxeda}{%
\subsubsection{3. Aunque ha calculado tablas de vida especificas por
raza y género, piense cómo se podría ``estimar'' el efecto de la raza
sobre el evento ``morir'' usando una función de la tabla de vida, es
decir, ¿cuál es el riesgo de muerte entre los hombres (si prefiere,
mujeres) blancos en comparación a los hombres (mujeres) negros? ¿Se
puede estimar ese efecto con estos datos? ¿Cómo lo
haría?}\label{aunque-ha-calculado-tablas-de-vida-especificas-por-raza-y-guxe9nero-piense-cuxf3mo-se-podruxeda-estimar-el-efecto-de-la-raza-sobre-el-evento-morir-usando-una-funciuxf3n-de-la-tabla-de-vida-es-decir-cuuxe1l-es-el-riesgo-de-muerte-entre-los-hombres-si-prefiere-mujeres-blancos-en-comparaciuxf3n-a-los-hombres-mujeres-negros-se-puede-estimar-ese-efecto-con-estos-datos-cuxf3mo-lo-haruxeda}}

hazard con covariables? regresion de cox

\end{document}
